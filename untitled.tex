\abstract{When conducting archaeological excavations of ancient cities, 3D reconstruction has become an important mechanism of documenting the findings and showing the results to general public in an accessible way. Most such reconstructions, however, mainly focus on visualising buildings and artifacts, while rarely simulating the actual people that populated the reconstructed city and aspects of their everyday life. Simulating such people and their lives in all their diversity is a costly and time-consuming exercise comparable in cost and efforts to development of a commercial video game, involving years of development and millions of dollars in funding. In this paper we present a novel approach that can significantly decrease the cost and effort required for simulating everyday life of ancient inhabitants of virtual cities, while still capturing enough detail to be useful in historical simulations. We show how it is possible to design a small number of individual avatars and then automatically simulate a substantially large crowd of virtual agents, which will live their lives in the simulated city, perform choirs and rituals as well as other routine activities that are consistent with their social status. The key novelty of our approach that enables simulating such sophisticated crowds is the combination of physiological needs - for generating agent goals,  emotions and personality - for choosing how to fulfil each goal and genetically informed propagation of appearance and personality traits - to propagate aspects of appearance and behaviour from a small sample of manually designed individuals to large agent groups of a desired size. The usefulness of our approach is demonstrated by applying it to simulating everyday life in the ancient city of Uruk, 3000 B.C.\footnote{See the prototype video at: \url{http://youtu.be/ZY_04YY4YRo}}}